\subsection{Campo}

\begin{itemize}
\item{En el constructor vac\'io de Campo creamos una Grilla de tres parcelas:
un Granero,
una Casa y un Cultivo, lo cual cumple el invariante del tipo Campo.}
	
	
\end{itemize}
 
\subsection{Drone}
 
\begin{itemize}
 
\item{En el constructor vac\'io de Drone ponemos 'posicionActual' en (0,0),
 'bater\'ia' en 100, 'enVuelo' false y 'trayectoria' la dejamos vac\'ia,
  lo cual cumple el invariante del tipo.} 

\item{En el constructor con par\'ametros de Drone, y teniendo en cuenta que no toma un Posici\'on para 'posicionActual' decidimos ubicarlo en el (0,0),
  lo cual cumple el invariante del tipo.} 
  
\item{En 'vueloEscalerado' decidimos, por simplicidad para la demostraci\'on, hacer expl\'icito el hecho de que toda trayectoria de dos posiciones o menos es escalerada.}
 
\end{itemize}


\subsection{Sistema}

\begin{itemize}

\item{En el contructor vac\'io de Sistema decidimos, por declaratividad, crear expl\'icitamente un Campo de 3 Parcelas y una lista vac\'ia de drones y asignarlos a las correspondientes variables de Sistema.}

\item{En volarYSensar, si el Drone llega a una Parcela en estado 'NoSensado', decidimos cambiarla a 'RecienSembrado', ya que la esecificaci\'on solo dice que deja de ser NoSensado. }

\end{itemize}

\subsection{Test}

\begin{itemize}
\item{En el test de 'despegar', por una consulta con un ayudante, al mover el Drone por fuera del enjambre su posic\'ion real no se modificaba, sacamos esa linea y agregamos la condic\'ion de que la posic\'ion que ocuparia sea valida. En el paquete agregamos la carpeta test con la modificac\'ion aqui descrita}
\end{itemize}